%Bonjour,
%
%Faisant suite à mon courrier du 30 mai, je vous transmets mes commentaires concernant les prochaines rencontres du Conseil d'École et du Comité Consultatif d'Établissement Élargi. Certains points ont déjà, me semble-t-il, été abordés, mais ils demeurent, à mon sens, essentiels. Par ailleurs, je n’ai pas observé beaucoup d’échanges récemment ; si un groupe Signal ou WhatsApp existe, pourriez-vous m’y ajouter ?
%
%Je souhaite encourager une implication plus active de l’établissement dans les concours scientifiques (mathématiques, physique, chimie), afin de sensibiliser davantage les élèves à ces matières. La participation collective, souvent requise, renforce l'esprit d'équipe et le sentiment d'appartenance. La préparation, au-delà de la compétition elle-même, est un levier pédagogique puissant. Si l’engagement des enseignants est crucial, des parents motivés pourraient au besoin également les soutenir pour alléger la charge de travail supplémentaire. Bien que ces concours soient principalement destinés aux collèges et lycées, certaines initiatives s’adressent également aux élèves du primaire. Parmi les activités notables, on peut citer :
%
%| Concours                                       | Classes  |  URL                      |
%|------------------------------------------------|----------|---------------------------|
%| Olympiades de Chimie                           | 2/1/T    | https://www.olympiades-chimie.fr/les-interlocuteurs/
%| Olympiade de Physique                          | 2/1/T    | https://www.olymphys.fr/public/index.php
%| Preparation Olympique Francaise de             | 5-1      | https://maths-olympiques.fr/
%| Mathematiques (POFM)                           |          |
%| Tournois Français des Jeunes Mathématiciennes  | 2/1/T    | https://tfjm.org/
%| et Mathématiciens (TFJM²)
%| Concours Alkindi : découvrez la cryptanalyse!  | 4/3/2    | https://concours-alkindi.fr/#/
%| « Filles, maths et informatique : une équation | 4/3/2    | https://filles-et-maths.fr/journees-filles-maths-informatique/
%| lumineuse ! »                                  |          |
%| Le Kangourou des Mathématiques                 | 6-T      | http://www2.mathkang.org/default.html
%| Le Koala des Mathématiques                     | Primaire | http://www2.mathkang.org/default.html
%| Le challenge Kids                              | Primaire | https://www.challenges-kids.fr/
%| Association Mathématique du Québec             | 5-T      | https://www.amq.math.ca/concours/
%
%Concernant les voyages scolaires, je propose que l'inscription à l'année implique un engagement clair à respecter les modalités définies par l'équipe enseignante : même moyen de transport pour tous, respect des règles collectives (ex. : pas de téléphone si non autorisé, etc.). Des dérogations médicales ou financières resteraient bien sûr possibles.
%
%Par ailleurs, je souhaite limiter l’usage des tablettes par les élèves. Le recours aux manuels ou exercices numériques entraîne une utilisation prolongée des écrans, ce qui n’est pas souhaitable pour leur santé et leur concentration.
%
%
%cdlt,
%Daniel
%
\section{Introduction}

Ce document a pour objectif de dresser un panorama des activités et concours d’excellence en lien avec le parcours scolaire. Il vise à offrir au Collège Stanislas une vision d’ensemble permettant d’encourager, de soutenir et de préparer les élèves à y participer activement.

La préparation à ces concours contribue au développement de compétences comparables à celles acquises dans les compétitions sportives : esprit d’équipe, rigueur et exigence. Dans cette perspective, le Collège souhaite instaurer un cadre favorisant une préparation collective et structurée à ces activités d’excellence.

\section{Mathématiques}

Cette section présente les concours liés aux Mathématiques. La Table~\ref{tab:math:ca} répertorie les initiatives canadiennes, tandis que la Table~\ref{tab:math:fr} recense les initiatives françaises.

Les modalités d'inscription peuvent être classées en trois catégories :
\begin{itemize}
 \item[\bf{I}ndividuelle]: cela signifie qu'un élève peut s'inscrire et participer au concours de manière individuelle. Cela n'exclut pas la possibilité d'une participation de la classe, mais permet simplement une initiative personnelle.
 \item[\bf{I}ndividuelle \bf{S}upervisée]: cela indique qu'un étudiant ou un groupe d'étudiants peut s'inscrire individuellement, MAIS l'inscription doit être gérée par un enseignant. Nous comprenons que le rôle de l'enseignant se limite à l'inscription et qu'il n'est pas impliqué dans le déroulement ultérieur des épreuves.
 \item[\bf{Ecole}]: cela signifie que l'inscription et la participation doivent être prises en charge par l'enseignant, impliquant généralement la participation de la classe.
\end{itemize}

La première observation est que la plupart de ces initiatives nécessitent l'implication de l'établissement scolaire. En effet, les concours permettant une participation individuelle non associée à l'établissement scolaire sont le Championnat international des jeux mathématiques et logiques (AQJM)~\cite{aqjm} et la Préparation Olympique Française de Mathématiques (POFM)~\cite{pqfm}.
Dans certains cas, l'inscription individuelle doit se faire par l'intermédiaire d'un enseignant. Dans d'autres cas, la classe doit être inscrite par l'enseignant. Par exemple, pour le Kangourou des Mathématiques~\cite{k} ou les concours organisés par le Centre d'Éducation en Mathématiques et en Informatique (CEMI)~\cite{cemi}, les candidatures individuelles ne sont pas autorisées~\cite{cemi-participation}.

\begin{quote}
Chaque année, les éducatrices et éducateurs commandent des concours du CEMI pour plus de 265 000 élèves à travers le monde, atteignant ainsi des élèves dans plus de 85 pays. Les élèves ne peuvent pas s’inscrire par eux-mêmes aux concours. L’inscription doit se faire par leur école. Si leur propre école ne participe pas, les élèves peuvent rechercher une autre école locale qui leur permettra de s'inscrire en tant que personne participante. Voici les étapes que les éducatrices et éducateurs peuvent suivre pour aider leurs élèves à participer aux concours du CEMI :
\end{quote}



\begin{table}[H]
\begin{tiny}	
\begin{tabular}{m{2.5cm}m{3.5cm}m{1cm}m{1cm}m{1cm}m{1cm}}\toprule
Organisme        & Concours                                          & Classe                    & Modalité & Inscription  &  Concours      \\ \midrule
\multirow{7}{2.5cm}{\centering Centre d'Education en Mathématique et en Informatique (CEMI)~\cite{cemi}, Archives~\cite{cemi-archive}} 
		 & Défi Team Up~\cite{tu}                            & 6\up{ième} - 4\up{ième}   & E & 28/05/26  & 01/06/26     \\ \cmidrule{2-6}
		 & Gauss~\cite{gauss}                                & 5\up{ième} - 4\up{ième}   & E & 23/04/26  & 11-22/05/26     \\ \cmidrule{2-6}
		 & Le Concours canadien de mathématiques par équipe~\cite{ctmc} & 3\up{ième} - T & E & 20/11/25 06/05/2026 & 8-9/05/26 \\ \cmidrule{2-6}
		 & Concours Fryer, Galois et Hypatie~\cite{fgh}      & 3\up{ième} - 1\up{ière}   & E & 05/03/26  & 01/04/26 \\ \cmidrule{2-6}  
		 & Concours Euclide~\cite{euclide}                   & T                         & E & 05/03/26  & 31/03/26 \\ \cmidrule{2-6}  
		 & Concours Pascal, Cayley et Fermat~\cite{pcf}      & 3\up{ième} - 1\up{ière}   & E & 10/02/26  & 25/02/26 \\ \cmidrule{2-6}  
		 & Concours canadiens de mathématiques de niveaux intermédiaire et supérieur~\cite{csimc}  & 3\up{ième} - T     & E & 16/10/25  & 13/11/26 \\ \midrule
Association Mathématique du Québec~\cite{amq}   &                    & 6\up{ième} - T            & E  &  ??         &  ??         \\ \midrule
Association Québecoise des Jeux Mathématiques (AQJM)~\cite{aqjm} & Championnat international des jeux mathématiques et logiques & CE2 - T & E, I & 15/01/25 & 15/01/16  \\ \midrule 
Olympiade de Mathématiques~\cite{olym-math} & Olympiade mathématique du Canada \& Olympiade junior (OMC/OMJC)  & & & \\ \midrule
\end{tabular}
\end{tiny}
	\caption{Mathématique (CA) \emph{I}, \emph{IS}, \emph{E} indique respectivement la possibilité d'une inscription \bf{I}ndividuelle, \bf{I}ndividuelle \bf{S}upervisée ou \bf{Ecole} \label{tab:math:ca}}
\end{table}


\begin{table}[H]
\begin{tiny}
\begin{tabular}{m{2.5cm}m{3.5cm}m{1cm}m{1cm}m{1cm}m{1cm}}\toprule
Organisme        & Concours                                          & Classe                    & Modalité & Inscription  &  Concours      \\ \midrule
\multirow{5}{2.5cm}{\centering Animath~\cite{animath}} & Preparation Olympique Francaise de Mathematiques (POFM)~\cite{pqfm} & 4\up{ième} - 1\up{ière}  & I, E & 12/09/25 & 12-18/09/25 \\ \cmidrule{5-6}
		       &	          &						                                  &  & 15/05/25 & 16-17/05/26 \\ \cmidrule{2-6}
		       & Concours Alkindi : découvrez la cryptanalyse !~\cite{alkindi} Archive~\cite{alkindi-archive}  & 4\up{ième} - 2\up{nd} & IS, E& 08/12/25 - 23/01/26 \\ \cmidrule{2-6}
		       & Tournois Français des Jeunes Mathématiciennes et Mathématiciens (TFJM²)~\cite{tfjm} & 2\up{nd} - T   & I, IS &  08/01/26 & 04-05/26   \\ \cmidrule{2-6}
		       & JOURNÉES FILLES, MATHS ET INFORMATIQUE : une équation lumineuse » (JFMI)~\cite{jfmi}  &  4\up{ième} - 2\up{nd} & I, IS, E & N/A & 20-21/10/25 \\ \midrule
	Le Kangourou des Mathématiques~\cite{k} & Kangourou & CE2 - T & N/A & E & 19/03/26 \\ \cmidrule{2-6}
			       & Koala     & CP - CE1 & N/A & E & 19/03/26 \\ \midrule
%\end{tabular}
\end{tabular}
\end{tiny}
	\caption{Mathématique (FR) \emph{I}, \emph{IS}, \emph{E} indique respectivement la possibilité d'une inscription \bf{I}ndividuelle, \bf{I}ndividuelle \bf{S}upervisée ou \bf{Ecole} \label{tab:math:fr}}
\end{table}



Les resources de POFM donnent beaucoup de liens interessants et en autre le site des Olympiades Suisse~\cite{om.ch} 


\section{Informatique}

\begin{table}[H]
\begin{tiny}
\begin{tabular}{m{2.5cm}m{3.5cm}m{1cm}m{1cm}m{1cm}m{1cm}}\toprule
Organisme        & Concours                                          & Classe                    & Modalité & Inscription  &  Concours      \\ \midrule
\multirow{2}{2.5cm}{\centering Centre d'Education en Mathématique et en Informatique (CEMI)~\cite{cemi}} 
		 & Concours canadien d'informatique (CCI)~\cite{cci} & 3\up{ième} - T & E & 12/02/26   & 18/02/26     \\ \cmidrule{2-6}
		 & Défi d’informatique Beaver (BCC)~\cite{bcc}       & CM2 - 2\up{nd} & E & 27/10/25   & 10-21/11/25  \\ \midrule 
Plateforme	 &  Le challenge Kids~\cite{kc}                      & Primaire       & N/A &  N/A         &  N/A           \\ \bottomrule
\end{tabular}
\end{tiny}
\caption{Activité Informatique}
\end{table}



\section{Physique / Chimie}

%| Organization | Concours                                                                | Classes FR | Individuel | Ecole | Date Inscription | Date Concours  |     lien |
%|--------------|-------------------------------------------------------------------------|------------|------------|--------|------------------|----------------|----------|
%|              |  Olympiades de Chimie                                                    | 2/1/T      |             https://www.olympiades-chimie.fr/les-interlocuteurs/
%|              |  Olympiade de Physique                                                   | 2/1/T      |             https://www.olymphys.fr/public/index.php

\section{Francais}

